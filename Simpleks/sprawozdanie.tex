\documentclass[a4paper,11pt]{report}
\usepackage[T1]{fontenc}
\usepackage[utf8]{inputenc}
\usepackage{lmodern}
\usepackage{graphicx}

\title{\textbf{Sprawozdanie z laboratorium\\Algorytm sympleksowy}}
\author{Adam Dąbrowski 184208}
\begin{document}
\maketitle
\chapter{Wprowadzenie}

Celem ćwiczenia było zapoznanie się z metodami rozwiązywania zadań programowania liniowego. Metoda sympleksów polego na iteracyjnym  polepszaniu rozwiązania.

\chapter{Realizacjia}

Program wylicza rozwiązanie przechodząc przez kolejne zmienne w równaniach, podobnie jak w metodzie eliminacji Gauasa zaczynamy ze złożonym problemem z nieznanym rozwiązaniem. Jednak po serii kroków, podczas których przepisujem system w równoważnej formie, posiadającej dodatkowe struktury uzyskujemy na tyle zmodyfikowany opis system, że rozwiązanie jest bardzo łatwe do uzyskania.


\chapter{Wnioski}
Metoda symbleksów jest to klasyczny sposób rozwiązywania problemów liniowych. Jest dość szybka w porównaniu do innych metod radzących sobie z tego typu problemami np. algorytmu elips. Mimo tego, że sama w sobie nie jest skomplikowana to jej implementacja nastręczyła mi wielu trudności. Domyślam się, że łatwiej by było zaimplementować ten algorytm w języku programowania bardziej zbliżonym do świata matematyki na przykład w środowisku Matlab.
\end{document}
